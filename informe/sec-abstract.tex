%\section {Resúmen}

\begin{abstract}
  En este trabajo práctico se analizará el retraso de vuelos,
  aplicando técnicas de \textbf{\textit{Métodos Numéricos}}\ y
  \textbf{\textit{Data Science,}}\ en particular
  \textbf{\textit{Regresiones Lineales de Cuadrados Mínimos}}\ sobre
  un conjunto de datos con el objetivo de poder ver información
  relevante y un modelo que nos permita predecir los retrasos
  aéreos. Luego se hará un análisis de la calidad de los resultados
  obtenidos mediantes la m\'etrica de \textbf{\textit{Root Mean
      Squared Error,}}\ conocido por su sigla como
  \textbf{\textit{RMSE.}}\
\end{abstract}

\begin{keyword}
\textit{RMSE} (\textit{Root Mean Squared Error}), \textit{OTP} (\textit{On-Time Performance}), \textit{KIP} (\textit{Key Performance Indicator}), \textit{Regresiones Lineales de Cuadrados Mínimos}, \textit{Retraso de vuelos}: \textit{Delays o Cancelación}
\end{keyword}

