\section{Introducción}

En estos últimos años se ha puesto el foco en hacer un análisis
exhaustivo en las activades periódicas de las empresas aeronáuticas,
con el fin de establecer si se han encontrado funcionando
correctamente.

\vspace{0.5em} El aumento de vuelos creció abruptamente dado la
expansión de aeropuertos a lo largo de todo Estados Unidos causando
dificultades de organización en la industria y congestión del tráfico
aéreo.

\vspace{0.5em} \cite{alonso2010mathematical} Según una investigación
de la Universidad Rey Juan Carlos, los problemas de congestión del
tráfico aéreo en los aeropuertos de Estados Unidos son cada vez más
graves y para mejorar la situación, las técnicas de Gestión de Tráfico
Aéreo tratan de anticipar y prevenir la situación de congestión que se
puedan producir, asignando retrasos a los vuelos o cancelándolos si
fuera preciso.

\vspace{0.5em} Debido a esto, en el siguiente trabajo práctico se
enfoca en la \textit{predicción de los retrasos de vuelos}. Para este
análisis, se aplicarán las técnicas de \textbf{\textit{Métodos
    Numéricos}}\ y \textbf{\textit{Data Science,}}\ en especial
\textbf{\textit{Regresiones Lineales de Cuadrados Mínimos.}}\ También
una forma de llevar adelante este enfoque será utilizando
\textbf{\textit{Key Performance Indicator (KPI)}} para evaluar la
puntualidad del transporte aéreo \textbf{\textit{On-Time Performance
    (OTP)}}

\vspace{0.5em} Se cuenta con un \textit{data set} comprendido con
cierta información relacionada a varios vuelos realizados en Estados
Unidos entre 1987 y 2008, incluyendo información de la compañia,
fecha, horarios planificados de arribo o partida, horarios de entrada
o salida, causa del delay, si fueron cancelados o no, su respectiva
causa, el tipo de avión utilizado, tiempo de vuelo, y tiempos de
\textit{taxi}.

\vspace{0.5em} Por último, en este trabajo práctico habrá una
experimentación usando la técnica de \textit{cross-validation}. Para
ello, vamos a particionar nuestro conjunto de datos y variar la
composición de la base de entrenamiento (\textit{training}) y las
observaciones consideradas como \textit{test} donde se tendrá en
cuenta cómo afecta a variables cuyo valor dependa de la cantidad de
observaciones tomadas.
