\section{Desarrollo}
Para estacionalidad en CML fijamos senoides con frecuencias y fases \footnote{Al no ser coeficientes CML no los permite resolver.} de acuerdo a nuestra intuición: buscamos frecuencias relacionadas a las estaciones como puede ser bimensualmente, trimestralmente, semestralmente y anualmente.
No granularizamos demasiado tales periodos para evitar \emph{overfittear}: no tiene sentido una estacionalidad cada 10 meses si no existen eventos importantes con esa frecuencia en el mundo real.
Para las fases decidimos usar aquellas que permiten que las ondas se inicialicen en puntos intermedios o en picos: $\frac{i\pi}{4}$.
Estas frecuencias anulan la necesidad de usar cosenos, que tampoco usamos para prevenir matrices singulares considerando los senos de distintas fases.
Viendo que las componentes polinomiales ajustaban mejor al entrenamiento pero a costas de predecir mucho peor decidimos dejar solamente hasta grado 1 para permitir cambios lineales.

Para validación cruzada usamos la librería de scikit-learn que dispone de métodos de partición para tal motivo de series de tiempo. La librería particiona, para $n$ splits, todas las secuencias con los primeros $\frac{i}{n+1}$ ($1\leq i\leq n$) registros para entrenamiento y el resto para predicciones.
